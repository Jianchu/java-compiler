\documentclass[a4paper, notitlepage]{report}
\usepackage{graphicx}

\usepackage{hyperref}
\usepackage{algpseudocode}
\usepackage{algorithm}
\usepackage{amsmath}
\usepackage{listings}
\usepackage[parfill]{parskip}
\usepackage[a4paper,left=3cm,right=3cm,top=2.5cm,bottom=2.5cm]{geometry}
\usepackage{titlesec, color}
%\usepackage[T1]{fontenc}
\definecolor{gray75}{gray}{0.75}
\newcommand{\hsp}{\hspace{20pt}}
\titleformat{\chapter}[hang]{\Huge\bfseries}{\thechapter\hsp\hsp}{0pt}{\Huge\bfseries}


\usepackage{etoolbox}
\makeatletter
\patchcmd{\chapter}{\if@openright\cleardoublepage\else\clearpage\fi}{}{}{}

\title{CS444/644 Assignment 1 Technical Report}
\author{Ho-Yi Fung, Jianchu Li, Zhiyuan Lin}
\date{\today}

\begin{document}

{\let\newpage\relax\maketitle}
\begin{abstract}

\end{abstract}


\tableofcontents



%\begingroup
%\let\clearpage\relax
\chapter{Introduction}

\chapter{Design}

\section{Name resolution}

In this stage of the program, it was decided we would follow closely the 7 steps specified in class:
\begin{enumerate}
	\item Constructing a global environment
	\item Resolving uses of syntactically identifiable type names
	\item Making sure class hierarchy is correct
	\item Resolving ambiguous names
	\item Resolving uses of local variables, formal parameters and static fields
	\item Type checking
	\item Resolving uses of methods and non-static fields.
\end{enumerate}

The first step involves construction of a symbol table and resolution of syntactically identifiable names.

To construct the class environment, we would first need to build the global class environment. The global class environment contains information of all the types (classes and interfaces) defined, including those from the standard library. Once the global environment is in place, we would then construct the scopes within each class. This process  needs to go over the declaration of all the types, methods, fields, and variables, and place these declarations in the proper scope.

The next step is to resolve the use of type names. This step deals with the following use of type names.
\begin{enumerate}
\item In a single-type-import
\item As super class of a type
\item As interface of a type
\item As a Type node in the AST, including:
	\begin{enumerate}
	\item In a field declaration or variable declaration (including the declaration of formals)
	\item As the result type of a method
	\item As the class name used in class instance creation expression, or the element type in an array creation expression
	\item As the type in a cast expression
	\item In the right hand side of the instanceof expression
	\end{enumerate}
\end{enumerate}
The above list is derived from the Java Language Specification, which defines syntactically identifiable type names in Section 6.5.1, and modified based on the Joos1W specification. 

The resolution type names would need to deal with both qualified names and simple names.As Java (and thus Joos) allows for use of fully qualified type name without import, and names cannot be partially qualified in Java, the resolution of qualified type name in Java is straightforward. One simply looks out the type from global environment. On the other hand, to resolve simple type names, we would need to look up the name, in the enclosing type declaration (the class or interface in which this type is used), in the single-type-import of the current compilation unit, in the package of the current class, and in the import-on-demand packages, exactly in this order. An error would be raised at this stage if no type under the name could be found. Besides if the name is resolved to more than one type (and thus is ambiguous), error would be thrown too. We also checks for prefix requirement in this process of resolving type name, that is for any qualified type name, its prefix should not be a valid type name in the class environment.

After type names are resolved, the class hierarchy needs to be built and examined.




\chapter{Implementation}
\label{implementation}
This chapter discusses the actual implementation of our program and the issues encountered during the implementation. This project was implemented in Java 8. \emph{Git} was used for version control and the code repository was hosted on University of Waterloo server with \emph{Gitlab}.

\section{Scanner}
\label{scanner_implementation}
% scanner implementation

%P94, Crafting
The implementation of scanner follows the DFA designed for valid tokens in our language. As explained in~\cite{fischer2009crafting}, a DFA can be implemented in two forms: \emph{table-driven}, or \emph{explicit control}. The table-driven approach, usually used by the scanner generators, utilizes an explicit transition table that could be interpreted by a universal driver program. The explicit control form, on the other hand, incorporates the transitions of the DFA directly into the control logic of the scanner program. In this project, we choose to hand-code the scanner in explicit control form for two reasons. First of all, incorporating the DFA transitions directly into control provides better performance. Secondly, the scanner produced in such manners is easier to debug and modify based on our requirements. The shortcoming of this approach is that, with the token definitions hard-coded into our program, the scanner could not be easily adapted for use elsewhere. This, however, is not a problem for the project. 
Another benefit of the explicit control implementation is that in the actual implementation there was no need to backtrack when running the Maximal Munch algorithm:
  the only states that were non-accepting and between accepting states were those associated with a block comment,
  and since an unfinished comment could not form a valid Joos program, the scanner would simply raise an lexical error when such situation arises.
  
%a slash (/) followed by an asterisk (*) could not form a valid Joos program, the scanner would simply raise an lexical error when such situation arises.

Another implementation detail was that since both strings and characters could contain escape characters, we would build a secondary DFA for escape characters, which was used as subroutine by both string and character scanning functions, rather than independently build it into both string and character scanning.

Each token created by the scanner would be assigned a type. For example, \emph{StringLiteral} is a token type, and each reserved keyword is a valid token type. We list every possible token type in an Java enum type called \emph{Symbol}. This enum also contains symbols that will later be used for building parse tree.

As discussed in Section~\ref{scanner_design}, we check for lexical errors such as non-ASCII characters. More specifically, the \emph{read()} function in the implementation checks for invalid characters every time a character is read. The scanning functions for strings, characters and comments would check for runaway strings, character, and comments respectively.
% Handling runaway Strings.

\section{Parser}

It was decided during implementation that rather than embedding the grammar directly into the code, we would keep it in a separate file so that,
  should it be required, the grammar could be regenerated without affecting the rest of the code. This approach follows the table-driven parser design discussed in Section~\ref{parser_design}.

 %Implementation of parse table in Java
Implementation of the parse table in Java was an important step. One of the key issue in the implementation is to make parse table access as fast as possible. The parse table was implemented practically as an array of hash maps (under a Java class called \emph{ParseActions}). An array was used because each state in the parse table is represented as an integer. Given the current state $s$ and the next symbol $t$, we first access the $s$-th hash map in the array, and then retrieve the action value with key $t$. If the hash map does not contain the key $t$, an error is raised. This implementation allows for action retrieval in time constant to the number of actions, therefore it is time efficient. 

Another important detail at this stage was the construction of parse tree. To make the code succinct, we have decide that the parse tree node would inherit from the \emph{Token} class used by the scanner, and they both share use of the \emph{Symbol} enum type, which contains both terminal and non-terminal symbols in our context free grammar.

\section{Weeder}

The implementation of the weeding phase was straight forward compared to the other modules. The weeder does a single traversal of the parse tree with a breadth-first search. And it checks for the errors discussed in Section~\ref{weeder_design}. 

\section{Abstract Syntax Tree }
The abstract syntax tree is constructed based on the parse tree from the previous stages. The parse tree is traversed once using recursive depth first search. 

The type hierarchy shown in figure~\ref{ast} was completely implemented. All AST nodes inherits directly or indirectly from the class \emph{ASTNode}. The functions that builds the AST are incorporated into the AST nodes. For example, the \emph{Statement} abstract class contains functions that transform a statement node in the parse tree into a \emph{Statement} node in AST, and this new statement node could be a \emph{WhileStatement}, \emph{ForStatement} node, etc.

With more than 40 nodes types and a complex type hierarchy, later stages such as type checking could have a lot of interacts with the AST nodes. To help better organize the code, the \emph{Visitor} design pattern was applied. The pattern makes it easy to encode a logical operation (a phase of compilation) for different AST node types under different methods in one file. An example of the \emph{Visitor} interface is provided in Figure~\ref{visitor}. The \emph{Visitor} interface will be implemented by future phases.

\begin{figure}

\begin{lstlisting}[language=Java, frame=single]  % Start your code-block

public interface Visitor {
	public int visit(Statement s);
	public int visit(Expression e);
	public int visit(Literal l);
	...
}
\end{lstlisting}
\caption{Example of Visitor Interface}
\label{visitor}
\end{figure}



Traversal Visitor

Type Linking Visitor

TopDecl visitor

Hierarchy Class


\chapter{Testing}
\label{testing}
All components in assignment 2, 3, and 4 are related to semantics analysis, so the main job of them is checking whether the subject program conforms to the Joos 1W semantics rules. Because of a great many rules in the specification, it's hard to test each rule's checker immediately after finishing it. So we created three JUnit 4 test suits to test the modules of each assignment separately in order to ensure the the robustness of our program.  

The JUnit 4 makes our testing procedure easier, so we followed the principle of unit testing, created a bunch of test cases for different situation. For example, in order to test that all accesses of protected class members are legal, we created test cases with different type hierarchy and access scenario, such that we can make sure the protected access checker works appropriately. Both positive and negative test cases, derived from the language specification, were used in this process. For example, each rule that should be checked by the type checker has its own test cases and was verified.

Because of the continuous nature of our project, the program was also tested incrementally, e.g. after each module (environment builder, type checker, etc.) was completed it was tested against all test cases that had been created even for previous modules. This approach ensures that any valid input could be run on all modules, while invalid input would fail at some point. 

In order to make sure that all semantics rules are checked by our compiler, we've also created test cases based on the example code from Joos 1W language specification~\cite{joos1w}.

The test cases from Marmoset were introduced and tested against after each assignment is finished. In each JUnit 4 test suit, the parameterized testing method was used for that purpose, such that we can make sure the program passes every test case. And also, after each addition/modification, those three test suits allow us test all modules again conveniently to prevent bugs being introduced to the program.   

The program so far successfully passes all the test cases devised by the team. This demonstrates the robustness and correctness of our implementation. 
%\endgroup

\chapter{Conclusion}

\bibliographystyle{ieeetr}
\bibliography{a4}

\end{document}