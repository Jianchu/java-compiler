\chapter{Testing}
\label{testing}
The unit testing method was used to ensure the robustness of our program, especially during implementation of scanner and parser. The JUnit 4 framework provides excellent support for this method. Following the principle of unit testing, each small unit in our program was tested separately. For example, the scanning of each token type was tested individually using parameterized tests. Both positive and negative test cases, derived from the language specification, were used in this process. For example, each error that should be detected by the weeder has its own test cases and was verified.

Because of the continuous nature of our project, the program was also tested incrementally, e.g. after each module (scanner, parser, etc.) was completed it was tested against all test cases that had been created even for previous modules. This approach ensures that any valid input could be run on all modules, while invalid input would fail at some point. 

We've also created test case based on the example code from  Joos 1W language specification~\cite{joos1w}. These tests ensure that the program satisfies the requirements for the language. 

Parse trees created by the parser was checked visually to ensure that it follows the grammar design. The weeder also provides confidence that the parse tree was constructed correctly. The effectiveness and correctness of the AST module was partly based on the correctness of parse tree due to their similarities. The AST will also be tested more extensively in later stages where it would be used as an input.

A JUnit test suite was also created so that after each addition/modification all modules could be tested again conveniently to prevent bugs being introduced to the program. After all parsing module of the program was completed, the tests cases from Marmoset was introduced and tested against. Parameterized testing was again used at this step to make sure the program passes every test case.

The program so far successfully passes all the test case devised by the team. This demonstrates the robustness and correctness of our implementation. 
