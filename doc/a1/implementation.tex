\chapter{Implementation}
\label{implementation}
This chapter discusses the actual implementation of our program and the issues encountered during the implementation. This project was implemented in Java 8. \emph{Git} was used for version control and the code repository was hosted on University of Waterloo server with \emph{Gitlab}.

\section{Scanner}
\label{scanner_implementation}
% scanner implementation

%P94, Crafting
The implementation of scanner follows the DFA designed for valid tokens in our language. As explained in~\cite{fischer2009crafting}, a DFA can be implemented in two forms: \emph{table-driven}, or \emph{explicit control}. The table-driven approach, usually used by the scanner generators, utilizes an explicit transition table that could be interpreted by a universal driver program. The explicit control form, on the other hand, incorporates the transitions of the DFA directly into the control logic of the scanner program. In this project, we choose to hand-code the scanner in explicit control form for two reasons. First of all, incorporating the DFA transitions directly into control provides better performance. Secondly, the scanner produced in such manners is easier to debug and modify based on our requirements. The shortcoming of this approach is that, with the token definitions hard-coded into our program, the scanner could not be easily adapted for use elsewhere. This, however, is not a problem for the project. 
Another benefit of the explicit control implementation is that in the actual implementation there was no need to backtrack when running the Maximal Munch algorithm:
  the only states that were non-accepting and between accepting states were those associated with a block comment,
  and since an unfinished comment could not form a valid Joos program, the scanner would simply raise an lexical error when such situation arises.
  
%a slash (/) followed by an asterisk (*) could not form a valid Joos program, the scanner would simply raise an lexical error when such situation arises.

Another implementation detail was that since both strings and characters could contain escape characters, we would build a secondary DFA for escape characters, which was used as subroutine by both string and character scanning functions, rather than independently build it into both string and character scanning.

Each token created by the scanner would be assigned a type. For example, \emph{StringLiteral} is a token type, and each reserved keyword is a valid token type. We list every possible token type in an Java enum type called \emph{Symbol}. This enum also contains symbols that will later be used for building parse tree.

As discussed in Section~\ref{scanner_design}, we check for lexical errors such as non-ASCII characters. More specifically, the \emph{read()} function in the implementation checks for invalid characters every time a character is read. The scanning functions for strings, characters and comments would check for runaway strings, character, and comments respectively.
% Handling runaway Strings.

\section{Parser}

It was decided during implementation that rather than embedding the grammar directly into the code, we would keep it in a separate file so that,
  should it be required, the grammar could be regenerated without affecting the rest of the code. This approach follows the table-driven parser design discussed in Section~\ref{parser_design}.

 %Implementation of parse table in Java
Implementation of the parse table in Java was an important step. One of the key issue in the implementation is to make parse table access as fast as possible. The parse table was implemented practically as an array of hash maps (under a Java class called \emph{ParseActions}). An array was used because each state in the parse table is represented as an integer. Given the current state $s$ and the next symbol $t$, we first access the $s$-th hash map in the array, and then retrieve the action value with key $t$. If the hash map does not contain the key $t$, an error is raised. This implementation allows for action retrieval in time constant to the number of actions, therefore it is time efficient. 

Another important detail at this stage was the construction of parse tree. To make the code succinct, we have decide that the parse tree node would inherit from the \emph{Token} class used by the scanner, and they both share use of the \emph{Symbol} enum type, which contains both terminal and non-terminal symbols in our context free grammar.

\section{Weeder}

The implementation of the weeding phase was straight forward compared to the other modules. The weeder does a single traversal of the parse tree with a breadth-first search. And it checks for the errors discussed in Section~\ref{weeder_design}. 

\section{Abstract Syntax Tree }
The abstract syntax tree is constructed based on the parse tree from the previous stages. The parse tree is traversed once using recursive depth first search. 

The type hierarchy shown in figure~\ref{ast} was completely implemented. All AST nodes inherits directly or indirectly from the class \emph{ASTNode}. The functions that builds the AST are incorporated into the AST nodes. For example, the \emph{Statement} abstract class contains functions that transform a statement node in the parse tree into a \emph{Statement} node in AST, and this new statement node could be a \emph{WhileStatement}, \emph{ForStatement} node, etc.

With more than 40 nodes types and a complex type hierarchy, later stages such as type checking could have a lot of interacts with the AST nodes. To help better organize the code, the \emph{Visitor} design pattern was applied. The pattern makes it easy to encode a logical operation (a phase of compilation) for different AST node types under different methods in one file. An example of the \emph{Visitor} interface is provided in Figure~\ref{visitor}. The \emph{Visitor} interface will be implemented by future phases.

\begin{figure}

\begin{lstlisting}[language=Java, frame=single]  % Start your code-block

public interface Visitor {
	public int visit(Statement s);
	public int visit(Expression e);
	public int visit(Literal l);
	...
}
\end{lstlisting}
\caption{Example of Visitor Interface}
\label{visitor}
\end{figure}

