\chapter{Introduction}

The Joos 1W language~\cite{joos1w} is a subset of the Java 1.3 language. The purpose of this project to create a compiler for the Joos 1W language in Java 8. In this assignment we've designed and implemented the modules in compiler that are responsible for lexical and syntactic analysis. More specifically, we implemented the \emph{scanner}, \emph{parser}, \emph{weeder} and \emph{abstract syntax tree} building module for the compiler. The \emph{scanner} takes as input the source code and convert them into a list of tokens. The \emph{parser} then reads the tokens and ensure that the program is grammatically correct. It also builds a parse tree in the process. The \emph{weeder} module runs after the parser to check for errors that are too complicated to encode in the grammar. Eventually we create an \emph{abstract syntax tree} based on the verified parse tree.

Chapter~\ref{design} discusses separately how the modules were designed and what issues were dealt with in each module. The details of our implementation are discussed in Chapter~\ref{implementation}. Chapter~\ref{testing} introduces the testing techniques and tools used to ensure the correctness of our program.